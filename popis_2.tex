
\catcode`<=13
\def<#1>{\hbox{$\langle$\it#1\/$\rangle$}}
\def\OPmac{\hbox{OPmac}}

\chap Původní stav

V této kapitole je uveden popis stroje před započetím práce. Tato bakalářská práce svým obsahem navazuje na bakalářskou práci Ondřeje Maslikiewicze  a rozšiřuje možnosti stroje o nastavení azimutu a elevace. Dále se tato práce bude zabývat tvorbou nové konstrukce a výběrem nové řídicí jednotky.

\sec Popis konstrukce 

Původní konstrukce byla tvořena dvěmi, protiběžně rotujícími kotouči, zásobníkem na badmintonové míčky a podávacím ramenem viz. Obrázek \ref[3D1]

\midinsert \clabel[3D1]{3D model prvního prototypu.}
\picw=13cm \cinspic pic/A_Sestava_badminton-3.jpg
\caption/f 3D model prvního prototypu.
\endinsert

Jako pohon kotoučů a podávacího ramena byli použity asynchroní motory firmy B\&R 8MSA2S.R0-42-Rev.D0. Tyto motory mají korouticí moment pouze 0,2~Nm, což celkem limituje možné zrychlení. Pro každý kotouč je použit jeden motor a tyto motory jsou pak pomocí virtuální převodovky synchronizovány.

\sec Řízení

K řízení bylo použito PLC firmy B\&R X20CP1485. Toto PLC mělo poměrně malou paměť RAM, z tohoto důvodu mohlo být ovládání pouze s rozlišením 320~x~240. K PLC byly přes sběrnici POWERLINK připojeny servořadiče ACOPOS 8V1010. Polohová zpětná vazba od motorů byla realizována pomocí resolverové karty AC122. 

Motory pohánějící kotouče jsou vzájemě synchronizovány pomocí virtuální převodovky s převodovým poměrem -1. 



\chap Prototyp 2

Druhý prototyp vychází z původního stroje, ke kterému přidává možnost změny elevace a azimutu. Konstrukce je opět tvořena rotujícími kotouči a podávacím ramenem. 

\midinsert  \clabel[3D2]{3D model druhého prototypu.}
\picw=.4\hsize
\centerline {\inspic pic/A_badminton+A+E-2.jpg \hfil\hfil \inspic pic/A_badminton+A+E-4.jpg }\nobreak
\centerline {a)\hfil\hfil b)}\nobreak\medskip
\caption/f 3D model druhého prototypu.
\endinsert

Pro pohon kotoučů, podavače a nastavení azimutu jsou opět použity motory B\&R 8MSA2S.R0-42-Rev.D0. K pohonu nastavení elevace je použit silnější motor B\&R 8LSA25.R0060D000-0 Rev.C4. Tento silnější motor má krouticí moment 0.6~Nm. I přes to, že byl na nastavení elevace použit silnější motor, tak se tento motor přibližně po minutě přehřál a servozesilovač spadl do chyby. Proto muselo být přidáno závaží, které vyvážilo celý mechanizmus. 

Motory jsou opět ovládány pomocí servozesilovačů ACOPOS 8V1010. K zařízení byl také přidán acess point díky čemuž lze zařízení ovládal pomocí technologie WiFi například z mobilního telefonu. Kompletní zapojení rozvaděče viz. Obrázek \ref[schema].

\midinsert \clabel[schema]{Schema zapojení rozvaděče.}
\picw=15cm \cinspic pic/schema_p2.JPG
\caption/f Schema zapojení rozvaděče.
\endinsert

Dále bylo třeba vyztužit celou konstrukci, jelikož byla nestabilní a při rychlé změně azimutu se celá rozkmitala. Při testování na kurtu byl objeven další problém, tím byl neodladěný podavač. Ten podával míčky takovým způsobem, že z nich kotouče otrhávaly peří a tím míčky ničily. Dále se občas stalo, že se opěrný drát rozkmital a podávací rameno se o něj zachytilo.  

\sec Dynamické parametry

K určení dynamických parametrů bylo provedeno několik experimentů. Cílem bylo zjistit maximální možnosti stroje a porovnat je s požaddavkama, které vyplývají z kapitoly \rfc[odkaz].

\secc Elevace

Z obrázku \ref[azimut1] vyplývá že pohyb z polohy $-45^\circ$ do polohy $45^\circ$ trval 0,8~s. Tento pohyb je téměř bez překmitu, bohužel při kratšim než $20^\circ$ má motor moc malý krouticí moment a tak nezvládne stroj ubrzdit a dochází k velkému překmitu, což je vidět na obrázku Obrázek \ref[azimut2]. V tomto případě trvá pohyb do ustálení přibližně 0,75~s. Proto je potřeba při takto krátkém přesunu snížit hodnotu zrychlení. Při snížení zrychlení o 60$ {ot}\over{min \cdot s^2}$

\midinsert \clabel[azimut1]{Graf pohybu z polohy $-45^\circ$ do polohy $45^\circ$.}
\picw=4cm \cinspic pic/azimut1.jpg
\caption/f Graf pohybu z polohy $-45^\circ$ do polohy $45^\circ$.
\endinsert

\midinsert
\line{\hsize=.5\hsize \vtop{%
      \clabel[azimut2]{Graf pohybu z polohy $90^\circ$ do polohy $100^\circ$.}
      \picw=4cm \cinspic pic/azimut2.jpg
      \caption/f Graf pohybu z polohy $90^\circ$ do polohy $100^\circ$.
   \par}\vtop{%
      \clabel[azimut3]{Graf pohybu z polohy $90^\circ$ do polohy $100^\circ$ se sníženým zrychlením.}
      \picw=4cm \cinspic pic/azimut3.jpg
      \caption/f Graf pohybu z polohy $90^\circ$ do polohy $100^\circ$ se sníženým zrychlením.
   \par}}
\endinsert

\secc Azimut

\secc Kotouče

\secc Podavač


\chap Prototyp 3

\sec Návrh pohonů

\sec Návrh konstrukce

\secc Kotouče

\secc Podavač

\secc Nastavení směru










\chap Nový řídicí systém

\sec Řídicí jednotka

\sec Řízení motorů






\chap Tréninkové programy

\sec Program 1

\sec Program 2

\sec Program 3





\chap Algoritmus řízení

\sec Výpočet doletu





\chap Uživatelské rozhraní




