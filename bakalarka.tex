% The documentation of the usage of CTUstyle -- the template for
% typessetting thesis by plain\TeX at CTU in Prague
% ---------------------------------------------------------------------
% Petr Olsak  Jan. 2013

% You can copy this file to your own file and do some changes.
% Then you can run:  pdfcsplain your-file

\input ctustyle   % The template is included here.
%\input pdfuni    % Uncomment this if you need accented PDFoutlines
\input opmac-bib % Uncomment this for direct reading of .bib database files 

\worktype [B/CZ] % Type: B = bachelor, M = master, D = Ph.D., O = other
                 % / the language: CZ = Czech, SK = Slovak, EN = English

\faculty    {F3}  % Type your faculty F1, F2, F3, etc.
            % use main language of your document here:
\department {Katedra řídicí techniky}
\title      {Rozšíření badmintonového stroje}
%\subtitle   {Šablona v plain\TeX{}u\nl pro sazbu 
 %            studentských závěrečných prací na ČVUT}
            % \subtitle is optional
\author     {Petr Kohout}
\date       {květen 2015}
\supervisor {Ing. Pavel Burget, Ph.D.}  % One or more supervisors
\studyinfo  {Program: Kybernetika a robotika \nl Obor: Systémy a řízení}  % Study programme etc.
%\workname   {Dokumentace} % Used only if \worktype [O/*] (Other)
            % optional more information about the document:
%\workinfo   {\url{http://petr.olsak.net/ctustyle.html}}
            % Title / Subtitle in minor language:
\titleEN    {Enhancement of the badminton machine}
%\subtitleEN {the plain\TeX{} template for theses at CTU}
            % If minor language is other than English
            % use \titleCZ, \subtitleCZ or \titleSK, \subtitleSK instead it.
\pagetwo    {}  % The text printed on the page 2 at the bottom.
\specification {\picw=16.6cm \cinspic pic/zadani_Kohout_Petr.pdf }
\abstractEN {
  This thesis deals with enhancement of the current badminton machine on possibility to change elevation and azimuth. It also deals with the issue of automated training programs and their implementation into the machine, user interface and testing. In the next section is analized torque on each axis and from these knowledge are derived requirements for engines to new prototype. Finally there are summarizes requirements for the new construction and suggested control and power unit for control new machine.
}
\abstractCZ {
   Tato práce se zabývá rozšířením stávajícího badmintonového stroje o možnost změny elevace a azimutu. Dále se zabývá problematikou automatických tréninkových programů, jejich implementací do stroje, uživatelským rozhraním a testováním. V další části je rozebrán točivý moment na jednotlivých osách a z těchto poznatků jsou odvozeny požadavky na motory právě navrhovaného prototypu. Nakonec jsou shrnuty požadavky na novou konstrukci, a navrhnuta řídicí a výkonová jednotka pro řízení nového stroje.
 
}           % If your language is Slovak use \abstractSK instead \abstractCZ

\keywordsEN {%
   Bachelor thesis, badminton, shuttlecock shooting, training, torque, PLC
}
\keywordsCZ {%
   Bakalářská práce, badminton, trénink, nadhazování míčků, točivý moment, PLC
}
\thanks {           % Use main language here
   Chtěl bych poděkovat vedoucímu práce Ing. Pavlu Burgetovi, Ph.D. za cenné rady a pomoc při řešení problémů.
   Dále bych rád poděkoval Ing. Milanovi Bartošovi, CSc. za výrobu konstrukce stroje a konzultace k mechanice.
   V neposlední řadě také děkuji Mgr. Michalovi Turoňovi za možnost testovat stroj na 
kurtu a za konzultace při tvorbě tréninkových programů. Dále bych rád poděkovat firmě B\&R za zapůjčení PLC X20CP3586 a powerlinkové karty V1.
   
}
\declaration {      % Use main language here
   Prohlašuji, že jsem předloženou práci vypracoval
   samostatně a že jsem uvedl veškeré použité informační zdroje v~souladu
   s~Metodickým pokynem o~dodržování etických principů při přípravě
   vysokoškolských závěrečných prací.

   V Praze dne \the\day. \the\month. \the\year % !!! Attention, you have to change this item.
   \signature % makes dots
}

%%%%% <--   % The place for your own macros is here.

\draft     % Uncomment this if the version of your document is working only.
%\linespacing=1.7  % uncomment this if you need more spaces between lines
                   % Warning: this works only when \draft is activated!
%\savetoner        % Turns off the lightBlue backround of tables and
                   % verbatims, only for \draft version.
%\blackwhite       % Use this if you need really Black+White thesis.
%\onesideprinting  % Use this if you really don't use duplex printing. 

\makefront  % Mandatory command. Makes title page, acknowledgment, contents etc.

\input uvod    % Files where the source of the document is prepared.
\input popis   % Full name is: uvod.tex, popis.tex, the suffix can be omitted.
\input prilohy

\bye
