% Lokální makra patří do hlavního souboru, ne sem.
% Tady je mám výjimečně proto, že chci nechat hlavní soubor bez maker,
% která jsou jen pro tento dokument. Uživatelé si pravděpodobně budou
% hlavní soubor kopírovat do svého dokumentu.

\def\ctustyle{{\tenss CTUstyle}}
\def\ttb{\tt\char`\\} % pro tisk kontrolních sekvencí v tabulkách
\chap Úvod

Badminton je velmi starý a v~poslední době populární sport. Jedná se o~sport, kdy protivníci, případně dvojice protivníků, pomocí rakety odpalují opeřený míček přes síť.
Hra probíhá na kurtu o~délce 13,4~m a šířce 5,18~m, u~čtyřhry je kurt široký dokonce 6,1~m. Výška sítě je 1,55~m.
\midinsert \clabel[picKurt]{Badmintonový kurt.}
\picw=13cm \cinspic pic/Badminton_court_3d_small.png
\caption/f Badmintonový kurt \cite[Kurt]
\endinsert
V~roce 1992 se badminton stal olympijským sportem a není divu, jedná se totiž o~nejrychlejší raketový sport. Rychlost smeče přesahuje i 300~km/h.
Aby však hráči dokázali podávat takovýto výkon, je zapotřebí tvrdého tréninku. Takový trénink ovšem není náročný pouze pro hráče, ale také pro trenéra. 
K~usnadnění tréninku slouží právě badmintonový stroj. Ten je určen k~nadhazování míčků, často v~takovém tempu, jaké by bylo pro trenéra dlouhodobě neudržitelné.

U~badmintonového stroje jsou kladeny vysoké nároky hlavně na rychlost podávání míčků a na dynamiku nastavení směru odpalu. Tyto nároky vyplývají z~požadovaných tréninkových programů viz kapitola \ref[programy].

Tato bakalářská práce svým obsahem navazuje na bakalářskou práci Ondřeje Maslikiewicze \cite[BPOndra] a rozšiřuje možnosti stroje o~nastavení azimutu a elevace. Dále se tato práce bude zabývat návrhem nové konstrukce a výběrem nové řídicí jednotky a akčních jednotek, novým uživatelským rozhraním a samotným algoritmem řízení.