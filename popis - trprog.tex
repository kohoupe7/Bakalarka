
\catcode`<=13
\def<#1>{\hbox{$\langle$\it#1\/$\rangle$}}
\def\OPmac{\hbox{OPmac}}

\chap Původní stav

V této kapitole je uveden popis badmintonového stroje před započetím úprav provedených v rámci této práce. 


\sec Popis konstrukce 

Původní konstrukce byla tvořena dvěmi, protiběžně rotujícími kotouči, zásobníkem na badmintonové míčky a podávacím ramenem. Celá konstrukce je vidět na Obrázku \ref[3D1].

\midinsert \clabel[3D1]{3D model prvního prototypu.}
\picw=13cm \cinspic pic/A_Sestava_badminton-3.jpg
\caption/f 3D model prvního prototypu.
\endinsert

Jako pohon kotoučů a podávacího ramena byli použity asynchroní motory firmy B\&R 8MSA2S.R0-42-Rev.D0. Tyto motory mají korouticí moment pouze 0,2~Nm, což celkem limituje možné zrychlení. Pro každý kotouč je použit jeden motor a tyto motory byly pak pomocí virtuální převodovky synchronizovány.

Kotouče byly vyrobeny ze silonu a měli průměr 198~mm. Na tyto kotouče byla, pomocí kovového mezikruží, natažena guma z duše k automobilu. Tímto pogumováním je zvýšeno tření mezi kotouči a míčkem. Bohužel tato guma nebyla přilepena a proto se při vysokých rychlostech nafukovala.

Podavač je tvořen zásobníkem na míčky, ve tvaru tubusu. Na konci je zůžení o které se míček zasekne. Pomocí dvou proti sobě umístěných podavačů je míček posunut o patro níže, kde je opět zůžení. V tomto zůžení je poté míček nabrán pomocí otočného ramene, které míček dopraví mezi kotouče viz Obrázek \ref[3Dpod]

\midinsert \clabel[3Dpod]{3D model podavače.}
\picw=13cm \cinspic pic/A_Sestava_badminton-2.jpg
\caption/f 3D model podavače.
\endinsert

První prototyp neměl nastavetelný azimut ani elevaci odpalu a proto byl pro opravdový badmintonový trénink téměř nepoužitelný. Celý tento prototyp sloužil pouze k ověření principu odpalu a technických možností konstrukce, jako je dostřel a možnosti podavače.

\sec Řízení

K řízení bylo použito PLC firmy B\&R X20CP1485. Toto PLC mělo poměrně malou paměť RAM, z tohoto důvodu mohlo být ovládací rozhraní pouze s rozlišením 320~x~240. Motory byly řízeny pomocí servozesilovačů ACOPOS 8V1010. Tyto servozesilovače měli dvě rozšiřující karty. Karta AC114 slouží pro komunikaci přes sběrnici POWERLINK. Pomocí této karty je zajištěna komunikace s PLC. Druhá použitá karta je resolverová karta AC122. Tato karta slouží k zajištění polohové zpětné vazby od motorů. 

Každý kotouč je poháněn samostatným motorem, tyto motory jsou vzájemě synchronizovány pomocí virtuální převodovky s převodovým poměrem -1. 



\chap Konkurenční řešení

Jako dnes už u téměř každého výrobku, i pro badmintonový stroj existuje konkurence.

\sec Knight Trainer

Prvním zástupcem konkurence je Knight Trainer viz. Obrázek \ref[knight]. Zařízení je montováno na stativu s nastavitelnou výškou od 1,98~m do 2,4~m. Stativ je montován na kolečkách, tím je zajištěna mobilita zařízení. Stroj dokáže vystřelit až 2,5 košíku za sekundu, s tím že zásobník pojme 50 košíků. Zařízení se ovládá pomocí infračerveného ovladače, kterým lze měnit frekvenci střelby a rychlost vystřeleného míčku. Zařízení neumí automaticky měnit směr ani elevaci. Knight Trainer je možné propojit s druhým zařízením a využít tak možnosti střílet na dvě různá místa a s dvojnásobnou frekvencí.

\midinsert \clabel[knight]{Knight Trainer \cite[knight].}
\picw=8cm \cinspic pic/knight.jpg
\caption/f Knight Trainer \cite[knight]. 
\endinsert


\sec Apollo Badminton Trainer

Další zástupce konkurenčních výrobků je Apollo Badminton Trainer zobrazený na obrázku \ref[apollo]. Zařízení je opět montováno na stativu s kolečkami, u kterého se dá nastavit výška v rozmezí 1,3~m až 1,8~m. Zásobník má kapacitu 250 košíků a zvládá vystřelit jeden košík za 1,2~s. Rozsah na natočení azimutu je $180^\circ$. Nastavení elevace zvládá v rozsahu $-15^\circ$ až $65^\circ$. Rychlost výstřelu je nastavitelná až do 130~mph. Nastavení směru lze provádět manuálně nebo automaticky podle některého ze zvolených tréninkových programů. Lze naprogramovat až 9 tréninkkových programů pomocí připojeného PC. Ovládání na kurtu potom probíhá pomocá bezdrátového ovladače. K tomuto stroji lze opět připojit druhý, čímž lze získat dvojnásobnou frekvenci odpalu tedy 0,65~s a dvojnásobnou kapacitu zásobníku.


\midinsert \clabel[apollo]{Apollo Badminton Trainer \cite[apollo].}
\picw=9cm \cinspic pic/apollo.jpg
\caption/f Apollo Badminton Trainer \cite[apollo]. 
\endinsert

\sec Další konkurence


\chap Prototyp 2


Druhý prototyp vychází z původního stroje, ke kterému přidává možnost změny elevace a azimutu. Konstrukce je opět tvořena rotujícími kotouči a podávacím ramenem. 


\sec Popis konstrukce 

Celá konstrukce je vidět na Obrázku \ref[3D2]. Celá původní konstrukce je nyní umístěna na kloubu. Tím je zajištěna možnost nastavovat elevaci výstřelu. Zařízení je také otočné, čímž jsme získali možnost měnit také směr, jakým poletí vystřelený míček.

\midinsert  \clabel[3D2]{3D model druhého prototypu.}
\picw=.4\hsize
\centerline {\inspic pic/A_badminton+A+E-2.jpg \hfil\hfil \inspic pic/A_badminton+A+E-4.jpg }\nobreak
\centerline {a)\hfil\hfil b)}\nobreak\medskip
\caption/f 3D model druhého prototypu.
\endinsert

Dále bylo třeba vyztužit celou konstrukci, jelikož byla nestabilní a při rychlé změně azimutu se celá rozkmitala. Při testování na kurtu byl objeven další problém, tím byl neodladěný podavač. Ten podával míčky takovým způsobem, že z nich kotouče otrhávaly peří a tím míčky ničily. Dále se občas stalo, že se opěrný drát rozkmital a podávací rameno se o něj zachytilo.  

\secc Ladění podavače

Po detailním zkoumání podavače bylo rozhodnuto, že tvar podávacího ramena je nevyhovující. Pro určení správného tvaru byla využita technologie 3D tisku. Bylo vyzkoušeno několik různých tvarů podávacího ramena viz. Obrázek \ref[podavace]. Nakonec byl vybrán tvar, který je uveden na obrázku \ref[podavace] vpravo. Tento tvar byl vybrán proto, že při nabírání košíku nedochází k zasekávání podavače a míček je poté dopraven mezi kotouče vždy stejně.

\midinsert \clabel[podavace]{Podávací ramena vytištěná na 3D tiskárně.}
\picw=10cm \cinspic pic/podavace.jpg
\caption/f Podávací ramena vytištěná na 3D tiskárně.
\endinsert


\sec Elektrické zapojení

Pro pohon kotoučů, podavače a nastavení azimutu jsou opět použity motory B\&R 8MSA2S.R0-42-Rev.D0. K pohonu nastavení elevace je použit silnější motor B\&R 8LSA25.R0060D000-0 Rev.C4. Tento silnější motor má krouticí moment 0.6~Nm. I přes to, že byl na nastavení elevace použit silnější motor, tak se tento motor přibližně po minutě přehřál a servozesilovač spadl do chyby. Proto muselo být přidáno závaží, které vyvážilo celý mechanizmus. 

Motory jsou opět ovládány pomocí servozesilovačů ACOPOS 8V1010. K zařízení byl také přidán acess point díky čemuž lze zařízení ovládal pomocí technologie WiFi například z mobilního telefonu. Kompletní zapojení rozvaděče viz. Obrázek \ref[schema].

\midinsert \clabel[schema]{Schema zapojení rozvaděče.}
\picw=15cm \cinspic pic/schema_p2.JPG
\caption/f Schema zapojení rozvaděče.
\endinsert

Stroj je napájen ze sítě 230~V/50~Hz. Servořadiče ACOPOS v sobě mají velký filtrační kondenzátor, který je zapojen proti zemnícímu vodiči. Při nabíjení tohoto kondenzátoru by tudíž docházelo k odpojování obvodu proudovým chráničem, proto je napájení realizováno přes oddělovací transformátor.

\midinsert \clabel[plc]{PLC B\&R X20CP3586.}
\picw=7cm \cinspic pic/X20CP3586.jpg
\caption/f PLC B\&R X20CP3586.
\endinsert

Za transformátorem je umístěn hlavní vypínač, kterým se vypíná napájení stroje. Na silové napájení jsou přes jističe připojeny silové vstupy servozesilovačů ACOPOS. Dále je na rozvod 230~V zapojena servisní zásuvka, do které je připojen nápájecí zdroj od acess pointu TP-LINK. Posledním zařízením připojeným na 230~V jsou zdroje stejnosměrného napětí 24~V. Zdroje jsou zde dva, jeden od firmy B\&R, kterým je napájeno PLC B\&R X20CP3586 (Obrázek \ref[plc]) a servozesilovače pro řízení azimutu a elevace. Druhým zdrojem je zdroj WAGO, ten napájí zbylé tři servozesilovače. Nulový potenciál obou zdrojů je spojený a přes rozpínací tlačítko TOTAL STOP je přiveden na vstupy ENABLE všech servozesilovačů. Servozesilovače komunikují s řídicím systémem pomocí sběrnice POWERLINK. 



\sec Dynamické parametry

K určení dynamických parametrů bylo provedeno několik experimentů. Cílem bylo zjistit maximální možnosti stroje a porovnat je s požadavkama, které vyplývají z kapitoly \ref[programy].

\secc Parametry PID regulátorů polohy

Dynamické parametry závisí na nastavení PID regulátoru polohy v servozesilovačích. Po přidání motorů pro nastavení azimutu a elevace bylo potřeba tyto parametry regulátorů nastavit. Parametry jsem se pokoušel ladit pomocí Zieger-Nocholsovy metody, ale vzhledem k omezenému rozsahu pohybu nebylo možno systém patřičně rozkmitat. Z tohoto důvodu jsem přistoupil k experimentálnímu nastavení konstant.

\secc Elevace

Z obrázku \ref[elevace1] vyplývá že pohyb z polohy $-45^\circ$ do polohy $45^\circ$ trval 0,8~s. Tento pohyb je téměř bez překmitu, bohužel při kratšim než $20^\circ$ má motor moc malý krouticí moment a tak nezvládne stroj ubrzdit a dochází k velkému překmitu, což je vidět na obrázku Obrázek \ref[elevace2]. V tomto případě trvá pohyb do ustálení přibližně 0,75~s. Proto je potřeba při takto krátkém přesunu snížit hodnotu zrychlení. Při snížení zrychlení o 60~$ {ot}\over{min \cdot s^2}$ se zmenší překmit a doba do ustálení se zkrátí přibližně na 0,5~s.

\midinsert \clabel[elevace1]{Graf pohybu z polohy $-45^\circ$ do polohy $45^\circ$.}
\picw=4cm \cinspic pic/elevace1.jpg
\caption/f Graf pohybu z polohy $-45^\circ$ do polohy $45^\circ$.
\endinsert

\midinsert
\line{\hsize=.5\hsize \vtop{%
      \clabel[elevace2]{Graf pohybu z polohy $90^\circ$ do polohy $100^\circ$.}
      \picw=4cm \cinspic pic/elevace2.jpg
      \caption/f Graf pohybu z polohy $90^\circ$ do polohy $100^\circ$.
   \par}\vtop{%
      \clabel[elevace3]{Graf pohybu z polohy $90^\circ$ do polohy $100^\circ$ se sníženým zrychlením.}
      \picw=4cm \cinspic pic/elevace3.jpg
      \caption/f Graf pohybu z polohy $90^\circ$ do polohy $100^\circ$ se sníženým zrychlením.
   \par}}
\endinsert

Z požadavků na tréninkové programy viz. Kapitola \ref[programy] vyplývá, že maximální požadovaná změna elevace je \rfc{doplnit data}. Tuto změnu stroj zvládne přibližně za \rfc{doplnit data}.

\secc Azimut

Dynamické parametry azimutové osy jsou o něco horší, celý stroj je totiž celkem težký a takto slabý motor nedokáže zajistit požadované zrychlení. Motor pro nastaveni azimutu má pouze 0,2~Nm což nestačí ani pro udržení polohy, proto motor dokáže pracovat maximálně minutu, poté již je teplota motoru kritická a servozesilovač vrátí chybu. Při přesunu o $25^\circ$ dostáváme hodnotu 1,1~s. Bohužel pohyb je s velkým překmitem, což je vidět na obrázku \rfc{dodat obrázek}.

\secc Kotouče

Požadavky na zrychlení kotoučů, které opět vyplývají z kapitoly \ref[programy]. V nejrychlejším režimu je potřeba měnit otáčky v rozsahu \rfc{rozsah otáček}. Tuto změnu otáček by v ideálním případě bylo potřeba stihnout za 0,3~s. Teoretické parametry vyplývají z momentu setrvačnosti kotouče a motoru a točivého momentu motoru. Moment setrvačnosti kotouče vyplývá z rozměrů viz Obrázek \ref[kotouc] a hmotnosti kotouče. K tomuto momentu setrvačnosti musíme připočíst moment setrvačnosti rotoru.

\midinsert \clabel[kotouc]{Výkres kotouče.}
\picw=14cm \cinspic pic/kotouc.jpg
\caption/f Výkres kotouče.
\endinsert

Moment setrvačnosti kotouče byl vypočítán pomocí programu Inventor a vyšel 2003,4~$kg\cdot mm^2$. Z dokumentace k motoru bylo zjištěno, že moment setrvačnosti rotoru je 60~$kg\cdot mm^2$. 
Maximální zrychlení je dáno vztahem \ref[momentsily].

\label[momentsily]
$$\alpha={{M}\over{J}} \eqmark $$
Kde $\alpha$ je úhlové zrychlení, M je moment síly a J je moment sertvačnosti. Moment síly dostáváme z dokumentace k motoru, kde jsme zjistili, že maximální stálý moment je 0,2~Nm a maximální špičkový moment je 0,8~Nm. Vzhledem k tomu, že rychlost otáčení měníme pouze jednou během jednoho výstřelu a v případě tohoto prototypu je maximální perioda výstřelů okolo 1~s. Proto můžeme počítat s momentem síly blížícím se maximálnímu špičkovému zatížení. V našem případě použijeme 0,7~Nm. Maximální zrychlení je tedy, podle vztahu \rfc{vztah}, xxxx.


\secc Podavač

Požadavky na podavač, které jsou opět dány v kapitole \ref[programy], jsou velmi náročné. V nejrychlejším režimu je potřeba střílet 3 míčky za vteřinu. Této rychlosti nejsme schopni dosáhnout, pokud budeme po každém výstřelu podavač zastavovat. Této rychlosti jsme schopni dosáhnout pouze v případě, že podavač roztočíme na potřebnou rychlost a vystřílíme všechny míčky najednou. V tomto režimu chvíli trvá než se podavač roztočí na požadovanou rychlost viz. Obrázek \rfc{obrázek}. Maximální zrychlení je opět dáno momentem setrvačnosti a točivým momentem motoru. 

\sec Měření přesnosti úderů




\chap Algoritmus řízení

V této kapitole je popsána struktura řídicího programu. Použité PLC se dá programovat v několika jazycích, já k programování zvolil strukturovaný text a B\&R Automation Basic. K programování je použit software Automation studio viz. Obrázek \ref[AS]. Dále je zde popsán výpočet teoretického doletu míčku a popis vývojového diagramu automatického režimu.

\midinsert \clabel[AS]{Vývojové prostředí Automation studio.}
\picw=15cm \cinspic pic/automationstudio.jpg
\caption/f Vývojové prostředí Automation studio.
\endinsert

\sec Struktura programu

Samotný program je rozdělen do balíčků a funkcí. Balíčků je velké množství a slouží zde pro inicializaci a řízení motorů. Všechny řídicí funkce jsou v balíčku Ostatní. Nejdůležitější funkcí je KontrolAut. Tato funkce se automaticky volá každých 100~ms a slouží k řízení odpalu při automatických tréninkových programech. Jedná se o stavový automat jehož struktura je uvedena na obrázku \ref[diagram]. V této funkci jsou také přednastaveny počáteční parametry jednotlivých programů. 

Ve stavu VYPNUTO je stroj v klidu a čeká na spuštění některého z programů. Po nastartování tréninkového programu přechází do stavu PRIPRAVA, kde dochází ke spuštění motorů a vynulování relativních počítadel polohy. Po úspěšném spuštění motorů následuje stav PRIPRAVAOK, kde je spuštěna virtuální převodovka a stavový automat se dělí na jednotlivé programy. Každý tréninkový program má tři stavy. První stav například PARAMETRY4 pro čtvrtý tréninkový program slouží k inicializaci startovních parametrů programu. Dalším stavem je NastaveniParametruP4, který vypočítá nové hodnoty parametrů pro další výstřel. Po omto stavu následuje stav PalP4 ve kterém dojde k odpalu míčku a přechází se zpět k NastaveniParametruP4. Tyto stavy jsou ale pouze u nově přidaných tréninkových programů, u náhodných programů, které byly obsaženy v prvním prototypu je pauze stav PARAMETRY a CHOD. Nakonec jsou zde stavy VYPINANI a PORUCHA. Stav VYPINANI slouží k vypnutí probíhajícího tréninkového programu. Dojde k zastavení všech pohonů a vypnutí motorů. Do stavu PORUCHA se stroj dostane při jakékoliv chybě pohonů, na diagramu znázorněno červenými přerušovanými šipkami.


\midinsert \clabel[diagram]{Vývojový diagram automatického řízení.}
\picw=15cm \cinspic pic/diagram.jpg
\caption/f Vývojový diagram automatického řízení.
\endinsert



\sec Výpočet doletu

Pro vetšinu tréninkových programů je naprosto klíčová znalost doletu míčku v závislosti na nastavení rychlosti kotoučů a elevace. Trajektorie letu košíku je balistická křivka. Bohužel nejsou známy parametry popisující chování míčku ve vzduchu, proto nelze spočítat dostřel přesně. Dalším problémem je rozdíl mezi různými míčky, při výpočtu by bylo potřeba zohlednit třeba i materiál nebo stáří míčku. V neposlední řadě je potřeba znát rychlost košíku při opuštění stroje. Tato rychlost se teoreticky rovná obvodové rychlosti kotoučů, ale z důvodu prokluzu míčku mezi kotouči bude vždy menší. Z těchto důvodů nikdy nespočítáme dostřel přesně a výpočet bude vždy jen přibližný. Experimentálně byl odvozen vztah \ref[dolet], který alespoň přibližně popisuje závislost doletu na elevaci a rychlosti kotoučů. Vztah je velmi zjednodušen, odvození přesné závislosti by bylo na samostatnou práci.

\label[dolet]
$$x={{L\cdot \cos(\varphi)}\over{2}}\cdot \ln \left (1+4\cdot \left ({{v_0}\over{v_{inf}}}\right )^2 \cdot \sin(\varphi)\right ) \eqmark $$

Kde x je vzdálenost ve které se míček nachází ve stejné výšce z jaké byl vystřelen v metrech. L je aerodynamická délka opět v metrech. Dále je zde $\varphi$ což je počáteční úhel elevece. V neposlední řadě vztah obsahuje počáteční rychlost košíku $v_0$ a maximální pádovou rychlost $v_{inf}$. Obě rychlosti se dosazují v $m\cdot s^{-1}$.


\chap Uživatelské rozhraní

Oproti puvodnímu prototypu bylo také rozšířeno uživatelské rozhraní. Původní vizualizace byla z důvodu omezené paměti použitého PLC pouze v rozlišení 320~x~240. I PLC použité v druhém prototypu má dostatek paměti pro spuštění vizualizace v rozlišení FullHD, zvolil jsem rozlišení nižší a to pouze 800~x~600. Takovéto rozlišení jsem zvolil hlavně proto, aby byo možné stroj ovládat i pomocí mobilního telefonu. Celá viualizace je vytvořena ve dvou jazycích a to v češtině a angličtině. Nastavení jazyka lze provést v servisní obrazovce. 
Na každé obrazovce je v pravém horním rohu tlačítko pro zobrazení náovědy k dané obrazovce a tlačítko s varováním na chybu stroje.

\sec Volba a nastavení tréninkového programu

Pro běžný provoz je použit automatický režim. V tomto režimu je na výběr několik automatických režimů, které je možno přepínat a nastavovat parametry pomocí vizualizace. Popis jednotlivých tréninkových programů je uveden v kapitole \rfc{kapitola}.

\midinsert \clabel[volbaProgramu]{Vizualizace - Výběr tréninkového programu}
\picw=11cm \cinspic pic/volba_programu.JPG
\caption/f Vizualizace - Výběr tréninkového programu.
\endinsert

Po vybrání tréninkového programu je možnost nastavit jednotlivé parametry programu jako je rychlost kotoučů, perioda odpalu nebo například počet míčků k odpalu viz Obrázek \ref[nastaveniProgramu]. 

\midinsert \clabel[nastaveniProgramu]{Vizualizace - Nastavení tréninkového programu}
\picw=11cm \cinspic pic/nastaveni_programu.JPG
\caption/f Vizualizace - Nastavení tréninkového programu.
\endinsert


\sec Chod tréninkového programu

Po spuštění některého z automatických režimů je po celou dobu tréninku zobrazena obrazovka, na které je zobrazen počet odpálených a zbývajících míčků. Z této obrazovky je také možno kdykoliv trénink ukončit.

\midinsert \clabel[chod]{Vizualizace - Chod programu}
\picw=11cm \cinspic pic/chod.JPG
\caption/f Vizualizace - Chod programu.
\endinsert

Z Obrázku \ref[chod] je vidět, že jsou zde zobrazeny i některé základní parametry režimu.

\sec Manuální režim a servisní obrazovka

Manuální režim slouží k ručnímu nastavení směru, roztočení motorů a k provedení ručního odpalu. Tento režim slouží například k měření parametrů stroje. Při správném nastavení stroje lze v manuálním režimu dostřelit kamkoliv v limitech stroje.

\midinsert \clabel[manual]{Vizualizace - Manuál}
\picw=11cm \cinspic pic/manual.JPG
\caption/f Vizualizace - Manuál.
\endinsert

Servisní obrazovka Obrázek \ref[servis] je určena k nastavení parametrů PLC a vizualizace. Je zde možnost přepnout mezi anglickým a českým jazykem, dále je zde možnost nastavit čas a síťové parametry PLC. Síťové parametry by měly být nastaveny s ohledem na nastavení AP TP-Link, aby nenastalo to, že by byl stroj v jiné podsíťi než AP, čímž by se znemožnila komunikace.
 
\midinsert \clabel[servis]{Vizualizace - Servis}
\picw=11cm \cinspic pic/servis.JPG
\caption/f Vizualizace - Servis.
\endinsert

\label[tocmoment]
\chap Analýza točivého momentu

Obsahem této kapitoly bude podrobná analýza točivého momentu na jednotlivých osách.

\sec Elevace

\sec Azimut

\sec Kotouče



\label[programy]
\chap Tréninkové programy

Požadavky na stroj byli konzultovány s badmintonovým trenérem Michalem Turoňem a podle těchto požadavků a možností stroje byly sestaveny jednotlivé automatické režimy. U každého z dále uvedených programů jsou uvedeny ideální parametry a parametry, kterých je schopen tento prototyp dosáhnout.

Tréninkové programy jsou rozděleny podle části hřiště do které jsou stříleny míčky. Jednotlivé zóny jsou znázorněny na obrázku \ref[zony].

\midinsert \clabel[zony]{Badmintonové hřiště - zóny dopadu míčků}
\picw=11cm \cinspic pic/zony.JPG
\caption/f Badmintonové hřiště - zóny dopadu míčků.
\endinsert

\sec Zadní část

Prvním automatickým režimem je střelba do zadní části hřiště. Při tomto režimu hráč procvičuje odehrávku míčů, které letí na zadní čáru. Takovéto míče mají celkem specifickou křivku letu, proto je nutné dodržovat elevaci v určitém rozsahu. Při testování na kurtu bylo určen jako ideální elevace $60^\circ$. 

Při střelbě do zadní části hřiště by měl stroj ideálně střílet přibližně jeden míček za \rfc{perioda} s v závislosti na obtížnosti programu. Dále by měl stroj střídat strany, na kterou míček letí. Z toho také vyplývá požadavek na rychlost změny azimutu, která by se měla stihnout dřív než bude vypálen další míček. 

Protože je potřeba střílet přes celé hřiště, tak je potřeba i poměrně vysoká rychlost otáček kotoučů. Při testování na kurtu je jednalo o obvodovou rychlost přibližně 45~$m\cdot~s^{-1}$. 

S druhým prototypem jsme schopni dosáhnout periody \rfc{perioda} hlavně kvůli nastavení azimutu, které je velmi pomalé. 

\sec Střední část 1

Druhý režim je zde pro střelbu do střední části hřiště. Hráč zde procvičuje odehrání míčů, které padají do hříště mezi první čárou a asi metro od konce kurtu. U míčků letících do této oblasti je také potřeba dodržovat určitou křivku letu, ale již není tak specifická jako u režimu střelby do zadní části proto je možná a dokonce žádoucí změna elevace. Změnou elevace je možné měnit prudkost střely.

Ideální perioda střel je v tomto režimu \rfc{perioda}, opět ale závisí na obtížnosti programu. Stroj by měl opět střídat strany a navíc také měnit úhel pod kterým míček odpaluje. Přestavení do nové polohy je opět potřeba stihnout mezi výstřely. 

V tomto režimu je klíčová změna stran, proto u druhého prototypu dosahujeme pouze \rfc{perioda} míčků za sekundu.

\sec Střední část 2
 
Dalším, celkem zajímavým, režimem je režim smeče. V tomto tréninkovém programu je potřeba stroj zvednout nad úroveň sítě. V tomto případě se střílí se zápornou elevací, čím je vlastně nasimulován smeč. Rychlostí kotoučů je možné měnit sílu smeče. 

Požadavek na frekvenci střelby je v tomto případě \rfc{perioda}. Dalším důležitým požadavkem je nejnižší možný dostřel, tím je myšleno jak blízko k síti dokáže stroj zasmečovat. Tento dostřel závisí na výšce, ze které stroj střílí a také na maximální velikosti záporné elevace. 

Stroj by měl mít možnost pokrýt co největší plochu a pokud možno střídat strany. U našeho prototypu je sice tento režim implementovám, ale protože stroj má celkem krátké kabely a spolu s rozvaděčem je celkem těžký. Z těchto důvodů by bylo velmi obtížné zvednout stroj nad síť. Skutečné parametry by ale měly mýt stejné jako u předchozího režimu.
 
\sec Na síť 

Režim třelby na síť je jedním z nejrychlejších režimů. Stroj nahazuje míčky těsně nad síť, případně těsně za síť s velmi vysokou frekvencí. Zároveň by měl stroj pokrýt celou šířku sítě. Ideálně by měl stroj střílet \rfc{perioda} míčků za sekundu a to buďto po dvou, to znamená dva najednou, změnit polohu a další dva. Další možností je velmi rychlá střelba s průběžnou změnou polohy.

Prototyp 2 tento režim zvládá bez problémů, pokud budeme měnit směr průběžně. Bohužel z důvodu slabého motoru na ose azimutu, zvládá tento režim pouze krátkodobě. 

\sec Na raketu

Poslední a nejrychlejší tréninkový program je nahazování na raketu. Tento režim nemusí být hrán přes síť. Režim spočívá v tom, že velmi rychle pálí míčky před sebe. Při nejvyžší obtížnosti by měly být odpáleny tři míčky za vteřinu. Takovéto rychlosti by nebyl trenér schopen dosáhnout. V tomto režimu není potřeba měnit směr a vše závisí pouze na podavači míčků. 





\chap Prototyp 3

Třetí a finální prototyp je prozatím ve stavu návrhu. V této kapitole jsou uvedeny informace týkající se návrhu nového prototypu. Nový stroj by měl mít podstatně menší hmotnost, aby byl schopen splnit požadavky uvedené v kapitole \ref[programy]. Dalším aspektem pro návrh nového stroje byla cena. Stroj by měl využívat levnější řídicí systém než dosavadní dva prototypy.

Ovládání by mělo být co nejvíce intuitivní a pokud možno nezávislé na jiných zařízeních jako notebook či mobilní telefon. 

\sec Návrh konstrukce

Jak již bylo řečeno tak jedním z hlavních požadavků na novou konstrukci je, aby byla co nejlehčí. Hmotnost je důležitá jak kvůlí snížení nároků na motory, tak i z důvodu přenosnosti stroje. 

\secc Kotouče

Konstrukce kotoučů bude, oproti prototypu 2, menší a to z důvodu snížení momentu setrvačnosti. Moment setrvačnosti kotouče se dá přibližně spočítat podle vztahu \ref[setrvacnost]. 

\label[setrvacnost]
$$J={{1}\over{12}} \cdot m \cdot r^2  \eqmark $$

Pokud průměr kotouče zmenšíme na 99~mm tedy na polovinu. Pokud bychom kotouče vyrobyli ze stejného materiálu, dostali bychom se na čtvrtinu hmotnosti. Po dosazení do vztahu \ref[setrvacnost] nám vyjde že se moment setrvačnosti kotouče zmenší na $1\over16$ původního momentu.

Problémem při zmenšení průměru je ten, že je potřeba zachovat maximální požadovanou obvodovou rychlost. U původního kotouče byla maximální rychlost motorů 4500~RPM což odpovídá obvodové rychlosti 46,653~$m\cdot s^{-1}$. Pro zachování této obvodové rychlosti při polovičním průměru je potřeba mít rychlost otáčení alespoň 9000~RPM. 

S ohledem na snížený moment setrvačnosti můžeme určit přibližné požadované parametry motoru \rfc{par mot}


Další změnou oproti druhému prototypu je, že mezera mezi kotouči bude menší než je průměr hlavičky a kotouče budou mezi sebou odpruženy. Tím se zajistí optimální přenos energie na košík aniž by se deformoval.

\secc Podavač

Na původním stroji byl podavač tvořen otočným ramenem, které podalo míček mezi kotouče. Tento podavač byl nevyhovující hlavně kvůli tomu, že mížek byl při podání mezi kotoučema v ležící poloze. V této poloze tak došlo k tomu, že byl míček protažen mezi kotouči i peřím a tím se výrazně snižovala jeho životnost. 

V novém stroji by měl být podavač řešen jiným způsobem. Nejspíše bude podávasí rameno ve tvaru kříže, který bude pod míčkem. Tím jak se bude kříž otáčet dojde k posunutí míčku mezi kotouče. 

\secc Nastavení směru

Jak již bylo uvedeno, je potřeba dosáhnout co nejlepších dynamických parametrů, proto je potřeba aby pro nastavení směru nebylo třeba pohybovat celým strojem. Z tohoto důvodu se nejspíše bude pohybovat pouze kotouči a podavačem míčků. Tento přístup má ovšem nevýhodu v omezeném rozsahu pohybu. Pro splnění požadavků na tréninkové programy je potřeba mít rozsah pohybu pro elevaci alespoň $-25^\circ$ až $60^\circ$. Požadovaný rozsah pohybu azimutu je alespoň $-60^\circ$ až $60^\circ$.


\sec Návrh pohonů

Při výběru pohoných jednotek jsem vycházel z analýzy točivého momentu viz. kapitola~\ref[tocmoment] a předpokládaných úprav. Z těchto informací byla odhadnuta maximální požadovanoá rychlost otáčení motoru a požadovaný točivý moment motoru. S ohledem na požadované vlastnosti motorů budou pro pohon kotoučů dva stejnosměrné motory případně brushless motory. \rfc{Pokud bych potřeboval natáhnout tak můžu udělat kapitolu o motorech} K nastavení směru se jeví jako nejvhodnější řešení, krokové motory. Krokové motory mají pro tuto aplikaci tu výhodu, že není potřeba externí enkodér k detekci polohy. Jako poslední pohoný prvek bylo potřeba vybrat pohon pro podavač míčků. Navržené konstrukci nejlépe vyhovuje upravené modelářské servo, kterému se odstraní zarážky aby se mohlo točit dokola. 

\sec Návrh řídicího systému

Řídicí systém použitý v předchozím prototypu byl sice funkční, ale také velmi drahý. Proto bylo potřeba zvolit nový systém 
řízení, což zahrnuje vybrat výpočetní jednotku, výkonovou část a navrhnout ovládací rozhraní.

\secc Výpočetní jednotka

Pro řízení všech funkcí stroje je naprosto vyhovující jednodeskový počítač Raspberry Pi 2 B viz. Obrázek \ref[raspi]. Jedná se o malý počítač přibližně velikosti platební karty. Tento počítač i přes svojí malou velikost a pořizovací náklady obsahuje vše potřebné pro běh operačního systému Linux.

\midinsert \clabel[raspi]{Raspberry Pi \cite[raspi]}
\picw=11cm \cinspic pic/raspi.jpeg
\caption/f Raspberry Pi \cite[raspi].
\endinsert

Tento konkrétní model je založen na čtyrjádrovém procesoru ARM Cortex-A7, který je taktován na frekvenci 900~MHz. Procesor je doplněn 1024~MB operační paměti, která je sdílena s grafickým procesorem VideoCore IV. Místo pevného disku je použita micro SD karta, která slouží jak pro zavedení operačního systému, tak pro ukládání uživatelských souborů. Počítač je dále vybaven ethernetovým adaptérem s přenosovou rychlostí 10/100 a čtyřmi USB porty.
\nl
\nl
Použitá verze Raspberry Pi dále obsahuje: 
\begitems
* 40 GPIO pinů
* HDMI (1080p)
* Zvukový výstup přes 3,5~mm konektor.
* CSI kamera port
* DSI display port
* Sběrnice UART, I2C a SPI
\enditems

Na Raspberry je nainstalován systém Raspbian, což je odnož Linuxové distribuce Debian. K počítači Raspberry existuje obrovská komunita, což má tu výhodu, že na internetu je k dispozici nepřeberné množsví návodů a příkladů téměř na vše. Většina těchto příkladů je psána v jazyce Python. 

\secc Výkonová jednotka

K počítači Raspberry Pi je potřeba připojit výkonové rozhraní pro řízení motorů. Pro tento účel byla vybrána právě vyvíjená deska pro řízení motorů od společnosti PIKRON. Tato výkonová část vycházi z desky MARS 8 od stejného výrobce. Hlavní výhodou je, možnost připojit téměř jakýkoliv motor. Je možnost připojit až 8 motorů a nezáleží na tom zda se jedná o brush, brushless nebo krokový motor. Tato výkonová část bude s řídicí částí komunikovat pomocí sběrnice UART.

\midinsert \clabel[mars]{Výkonová jednotka.}
\picw=11cm \cinspic pic/mars.jpg
\caption/f Výkonová jednotka.
\endinsert

Zařízení se skládá z výkoné části, jejíž vývoj momentálně probíhá. Druhou částí je řídicí část, jejímž srdcem je procesor ARM Cortex-M3 a hradlové pole FPGA Spartan 6. Komunikační možnosti této elektroniky jsou rovněž bohaté. Již zmíněný UART doplňuje převodník FTDI, který emuluje rozhraní USB. Dalšími možnostmi komunikace je ethernetové rozhraní, případně sběrnice CAN, I2C či SPI. Veškeré datové kanály jsou opticky odděleny.

\secc Ovládání a uživatelské rozhraní


\chap Závěr




